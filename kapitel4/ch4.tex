%%
%% Beuth Hochschule für Technik --  
%%
%% Kapitel 3 - Desktop App
%%
%%	

\chapter{Desktop App}
Zu der Android App soll es parallel eine App für den Desktop geben. Diese wurde mit mit JavaFX programmiert.
\section{Was ist JavaFX ?}
JavaFX ist eine Java Spezifikation die als Hauptkonkurrenten Adobe Flash und Microsoft Silverlight hat. Ein positiver Punkt ist der Lauffähigkeit auf diversen Geräten wie z.B. Mobilfunk, Desktop-Computern, Embedded Geräten und Blu-ray Geräten. Die Programmierung findet ganz normal wie in Java statt. Die dazu gehörigen Bibliotheken werden seit der Java SE Runtime 7 Update 6 automatisch mit installiert. Es ist unter anderem auf die Grafikprogrammierung ausgelegt. Dadurch lassen sich Grafische Elemente schnell programmieren und mit CSS gestalten.(Quelle: \cite{bib.jFXRaspPi})\newline
Ein sehr bekanntes Embedded Gerät wofür es auch JavaFX gibt ist das Raspberry Pi.
\section{Struktur und Aufbau der App}
